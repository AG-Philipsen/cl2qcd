\documentclass[a4paper]{article}

\usepackage[utf8]{inputenc}
\usepackage[T1]{fontenc}

\usepackage{amsmath}

\newcommand{\psibar}{\ensuremath{\overline\psi}}
\newcommand{\vev}[1]{\ensuremath{\left<#1\right>}}
\newcommand{\ii}{\ensuremath{\text{i}}}

\begin{document}
\author{Lars Zeidlewicz}
\date{Sep-19-2011}
\title{Flavour Multiplet Correlators from Point Sources}
\maketitle

\section{Conventions}

We consider correlators of type
\begin{equation}
C_\Gamma = \vev{\psibar_x \Gamma \tau^+ \psi_x \psibar_0 \Gamma \tau^- \psi_0}
\end{equation}
where $\Gamma$ is some combination of Dirac matrices that determines the quantum numbers (along with $\tau^\pm$ which causes the correlator to be a iso-doublet, i.\,e.\ to describe a charged meson state). The evaluated correlator reads
\begin{equation}
C_\Gamma = - \text{Tr} \left( S_u^\dagger(0,x) \gamma_5 \Gamma S_u(0,x) \Gamma \gamma_5 \right)\;.
\end{equation}
Note the position of the source $x_0 = 0$ can be changed arbitrarily.

We consider the following naming scheme:

\begin{tabular}{|lll|}
\hline
operator                           & name                      & PDG label\\
\hline\hline
$\Gamma = 1$                       & scalar meson (SC)         & $a_0^\pm$ \\
$\Gamma = \gamma_5$                & pseudo scalar meson (PS)  & $\pi^\pm$ \\
$\Gamma = \gamma_j\gamma_5$        & vector meson (V)          & $\rho^\pm$ \\
$\Gamma = \gamma_j\gamma_4\gamma_5$& axial vector meson (A)    & $b^\pm$\\
\hline
\end{tabular}

For the vector type mesons the index $j$ can run over all three spatial directions so that there are in fact three correlators each.\footnote{Note: In our output we give the average of those three correlators and each of them separately. This is different to tmLQCD combined with their contractions code where the sum instead of the average of these correlators is calculated.}

Note that we are a bit sloppy about prefactors and signs: Actually the operators should be dressed by appropriate factors of $\ii$ and $1/2$. The imaginary prefactors contribute to the overall sign of the correlator which is -- however -- fixed to be positive. The factors of $1/2$ mean that our correlators should differ from frequent conventions (see e.\,g.\ ETMC) by an overall factor of 4.\footnote{We cannot find this factor. Maybe tmLQCD with their contractions code does not include it either?}

In the same manner, we neglect a sign from adjoining the vector operator. All those signs can be fixed afterwards.


\section{Solving the Dirac Equation}
Numerically, we find the solution $\phi$ to the Dirac equation,
\begin{equation}
M_{ij}(x,y) \phi_j(y) = \eta_i(x|z,k)\;.
\end{equation}
The vector and matrix indices ($i,j,k$) run over colour-spin space, i.\,e.\ $i=1,\ldots,12$. We consider point sources, i.\,e.\
\begin{equation}
\eta_i(x|z,k) = \delta_{i,k} \delta_{x,z}
\end{equation}
and assume $z=0$ for the following (generalisation is straight forward\ldots).  With $S=M^{-1}$ we thus have
\begin{equation}
\phi_l^{(k)}(x) = S_{lk}(x,0)\;.
\end{equation}
The superscript of $\phi$ refers to the source, $j=1,\ldots,12$.

For the correlator we can now write
\begin{equation}
C_\Gamma(x,0) = - \left(\phi_j^{(i)}\right)^* \left(\gamma_5\Gamma)\right)_{jk} \phi_k^{(l)}(x) \left(\Gamma\gamma_5\right)_{li}\;.
\end{equation}
The commonly calculated correlators $C_\Gamma(z)$ and $C_\Gamma(t)$ are connected to the above formula via the relations\footnote{Note that compared to the contractions code for tmLQCD we average $C(z)$ by $N_\sigma^2N_\tau$ instead of $N_\sigma^3$.}
\begin{subequations}
\begin{align}
C_\Gamma(z) & = \frac{1}{N_\sigma^2 N_\tau} \sum_{x,y,t} C_\Gamma( (x,y,z,t), 0 )\;,\\
C_\Gamma(t) & = \frac{1}{N_\sigma^3} \sum_{x,y,z} C_\Gamma( (x,y,z,t), 0 )\;,\\
\end{align}
\end{subequations}

Hence, the pseudo-scalar with $\Gamma\gamma_5=1$ can be calculated as
\begin{equation}
C_{PS}  \sim \sum_i \left| \phi^{(i)} \right|^2\;.
\end{equation}
For the other correlators an explicit evaluation of the Dirac structure is necessary. 


\section{Mathematica File}
The Mathematica file \verb+correlators.nb+ allows to extract the necessary structure, see comments there. The solutions $\phi$ are encoded as a matrix: $\phi^{(i)}_j = \phi_{ij}$. If you use the correlator function, remember that the matrix input needed corresponds to $\Gamma\gamma_5$; e.\,g.\ the pseudo-scalar structure is obtained via
\begin{verbatim}
correlator[unit,phi]
\end{verbatim} 
Unfortunately, Mathematica is not intelligent enough (or my knowledge of that program is too limited) in order to simplify the output so that this needs to be done ``manually''. All correlators should be sums/differences of either the squarnorm or the real part of entries of $\phi$. Note that we have only encoded the Dirac-structure. Summation of colour space is still needed (but trivial since we always consider colour singlet correlators).


\section{References}
I neglect extensive references here. You can find them in my PhD thesis. A general collection of ETMC conventions is found in Comput.\ Phys.\ Commun.\ 179 (2008) 695-715 (arXiv:0803.0224).



\end{document}
