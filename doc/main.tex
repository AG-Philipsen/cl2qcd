\documentclass[a4paper,11pt]{article}

\usepackage[utf8]{inputenc}
\usepackage[T1]{fontenc}

\usepackage{rotating}

\usepackage{amsmath}

\title{Project documentation:\\
       \textbf{HMC with OpenCL for hybrid manycore systems}
}

\author{Christopher Pinke \and Lars Zeidlewicz}

\date{\today}

\bibliographystyle{unsrt}

\newcommand{\file}[1]{\texttt{#1}}
\newcommand{\define}[1]{\texttt{#1}}
\newcommand{\ctype}[1]{\texttt{#1}}
\newcommand{\function}[1]{\texttt{#1}}
\newcommand{\url}[1]{\texttt{#1}}

\begin{document}
\maketitle
\begin{abstract}
We want to write a complete HMC simulation programme with Wilson-type quarks (twisted mass action) using OpenCL for a hybrid structure consisting of CPUs and GPUs.
\end{abstract}

\tableofcontents

\newpage

\section{General thoughts -- structure}
This will be a lot of work\ldots

Basically, the programme splits into two main parts: host-code and device-code. Host-code is standard C++ and provides the OpenCL-interface and master code for the OpenCL kernels. Furthermore, in the end, it is planned to have a completely working CPU-HMC, too. Since it is meant to run on a single node, the use of OpenMP is intended. The CPU-HMC might be needed to implement the hybrid approach in a later step.

The device code is the collection of all OpenCL-kernels. This should finally constitute a complete HMC. For now, we target at a heatbath implementation. The pure gauge heatbath will provide a good testing ground for a) benchmarks (and performance optimisation), b) the hybrid strategy. At this stage, we also want to learn what the optimal communication between host and device is, i.\,e. when do we need to transfer what information.

Data types for the C++ host and OpenCL device codes are not the same. Therefore, the C++ code also needs to provide transfer functions. For the gaugefield and SU($N_c$) types this should be finished. For the spinor part, since its not needed yet, no OpenCL types have been defined so far.

On the host, there are plaquette and Polyakov functions that have been tested against existing configuration files. It is an open question whether we want to have those measurements on the device, too. The alternative would be to measure the gauge observables only when the gaugefield is transferred back to the host.

ILDG format configurations (as from Carsten Urbach's {\bf reference\_has\_to\_be\_added\_by\_Lars} code) can be read. We still need the according functions to write a config to a file.

\section{Algorithm}
In this section we describe the algorithm that is implemented in the programme. Ultimatively, we want to have an HMC with CG-solver (or maybe BiCGStab), even-odd preconditioning, leap-frog integration and possibly mass preconditioning {\bf references\_to\_be\_given\_by\_Lars}. That seems to be the standard choice of algorithm.

\subsection{Heatbath}
Right now only the heatbath algorithm is implemented. There is nothing fancy about it but it corresponds to the standard description given in {\bf reference\_to\_be\_given\_by\_Christopher}.

\section{Options and I/O}
\subsection{Input-file}
All informations on run-time options are passed to the programme in an input file. The name of that file is the only command line option to be given when calling the executable. This file contains one line per option/variable with the format\\
\verb+OPTIONNAME = +\textit{option}.\\
If one option name occurs several times, the last instance is used. It is possible to have comment lines starting with ``\#''.
The list of options and variables is given in table~\ref{tab:options}.

\begin{sidewaystable}
%\begin{table}
\begin{tabular}{llcc}
\verb+OPTIONNAME+ & description                        & possible values              & default \\
\hline
\verb+kappa+      & hopping parameter $\kappa$         & $0<\kappa<\infty$            & 0.125 \\
\verb+beta+       & lattice coupling $\beta$           & $0<\beta<\infty$             & 4 \\
\verb+mu+         & twisted mass parameter $\mu$       & $0<\mu<\infty$               & 0.006 \\
\verb+cgmax+      & maximum number of CG iterations    & $1<\texttt{cgmax}<\infty$    & 1000 \\
\verb+prec+       & precision of gaugefield (from file)& 32, 64                       & 64 \\
\verb+startcondition+ & start condition                & hot, cold, continue          & cold \\
\verb+thermalizationsteps+ & thermalisation steps      & $0\le\texttt{thermalizationsteps}<\infty$ & 0 \\
\verb+heatbathsteps+  & heatbath steps                 & $0<\texttt{heatbathsteps}<\infty$ & 1000 \\
\verb+writefrequency+  & frequency to write gaugeobs.  & $1<\texttt{writefrequency}<\infty$ & 1 \\
\verb+saveconfigs+  & if to save gauge configs at all  & true, yes; false, no & FALSE \\
\verb+savefrequency+  & frequency to save gauge files  & $0<\texttt{savefrequency}<\infty$ & 100 \\
\verb+sourcefile+  & name of gauge config file         & string                       &  not defined
\end{tabular}
\caption{List of all possible options and variables that can be passed to the programme in the input file.} \label{tab:options}
%\end{table}
\end{sidewaystable}

\subsection{Configurations}
Configuation files can be read from ILDG format. So far, no configurations can be saved to a file. {\bf some\_more\_information\_should\_be\_provided\_by\_Christopher}

\subsection{Compiler options}
\subsubsection{\define{RECONSTRUCT\_TWELVE}}
If \define{RECONSTRUCT\_TWELVE} is defined, SU(3) matrices are stored using 12 floating point numbers of type \ctype{hmc\_float}. They represent the first two rows of the matrix. The last row is reconstructed~\cite{Clark:2009wm}. The 12 numbers are stored using a single index $n=i+(N_c-1)j$ for the components $U_{ij}$. The reconstruction is done in \file{operations.cpp}. You have 
\[ j = \text{int} \left(\frac{n}{N_c-1}\right) \]
and
\[ i = n - (N_c-1)j\;. \]
Thus the numbering works as follows:
\[ \left(\begin{array}{rrr} u_{00} & u_{01} & u_{02} \\ u_{10} & u_{11} & u_{12} \\ u_{20} & u_{21} & u_{22}\end{array}\right)
=
\left(\begin{array}{rrr} u_{0} & u_{2} & u_{4} \\ u_{1} & u_{3} & u_{5} \\ c_0 & c_1 & c_2 \end{array}\right)
\]
The reconstruction of the elements $c_\text{ncomp}$ of the matrix \ctype{in} is done with the function \function{reconstruct\_su3(hmc\_su3matrix* in, int ncomp)}.

\subsubsection{\define{SOURCEDIR}}
\define{SOURCEDIR} is passed to the OpenCL \ctype{clprogram} in order to find the OpenCL kernel files.
\subsubsection{\define{\_OPENMP}}
\define{\_OPENMP} can be used to activate OpenMP.
\subsubsection{\define{\_USEDOUBLEPREC\_}}
\define{\_USEDOUBLEPREC\_} switches to double precision floating point numbers on the OpenCL device.
\subsubsection{internal switches}
\begin{itemize}
\item 
\define{\_INKERNEL\_} allows to check whether OpenCL kernel code or C++ host code is meant to be compiled (some \file{globalheaders.h} and \file{types.h} are included by both of them).
\end{itemize}

\subsection{Reading the gaugefield}
\file{readgauge.h}, \file{readgauge.cpp} provides a class \ctype{sourcefileparameters} which reads in all information from a given tmlqcd file concerning gauge-configurations. Not fully implemented is the read-in of fermion-propagators. 

The reading-routine is based on a stand-alone-program that was written in C. The metainformations about the data are saved in XML- and XLF-format (see also appendix), while the actual data is saved binary. To read XLF and XLM files the LIME- (source) and XLM-libraries are used, which are only available for C. The routines are not optimized regarding speed or memory-usage since the reading of the data should only take place at the beginning of the programrun.

The reading of the metainforamtions was specifically tested for two tmlqcd-files created with different version of the hmc (5.1.1 and 5.1.5). Everything is read and saved into particular variables. If there are fermion-informations in the sourcefile, they are saved successively for each fermion. Here, only informations about the first are stored, all others are not read. This has not been implemented because there is most likely no need to read in propagators.

The binary data is saved into a field which is a parameter of the routine (as a pointer). The precision in which the data was saved is extracted before during the metainformation gain. ILDG is always stored big endian, so the routine checks the local endianness with htons() to get the right order of the bytes.





\section{Functions and constants}
\subsection{\file{globaldefs.h}}
Definitions in \file{globaldefs.h} should be available programme-wide.
\begin{itemize}
\item \define{NC}=3
\item \define{NSPIN}=4
\item \define{NDIM}=4
\item \define{NSPACE}, \define{NTIME}: The spatial and temporal extent is defined at compile time in order to have fixed for-loop lengths. These two definitions are passed to the OpenCL kernels with their compile-time values.
\item \define{VOLSPACE} = \define{NPACE*NSPACE*NSPACE}
\item \define{VOL4D} = \define{VOLSPACE*NTIME}
\item \define{PI} = 3.14159265358979
\item \define{su2\_entries} = 4
\item \define{START\_FROM\_SOURCE}=2, \define{COLD\_START}=0, \define{HOT\_START}=1
\end{itemize}

\subsection{\file{hmcerrs.h}}
\file{hmcerrs.h} defines the error codes as \verb+typedef int hmc_error+:
\begin{itemize}
\item \define{HMC\_SUCCESS}
\item \define{HMC\_STDERR}
\item \define{HMC\_FILEERROR}
\item \define{HMC\_OCLERROR}
\item \define{HMC\_XMLERROR}
\item \define{HMC\_UNDEFINEDERROR}
\end{itemize}

\subsection{\file{types.h}}
\file{types.h} contains typdefs. There are differences between host and kernel code. For both, the basic floating point type is \ctype{hmc\_float}
\begin{verbatim}
#ifdef _USEDOUBLEPREC_
typedef double hmc_float;
#else
typedef float hmc_float;
#endif
\end{verbatim}
With this type \ctype{hmc\_one\_f}=1 is a global constant. For both host and kernel, there is a complex type \ctype{hmc\_complex} with members \ctype{re} and \ctype{im} for real and imaginary part that is based on \ctype{hmc\_float}. \ctype{hmc\_complex\_one}, \ctype{hmc\_complex\_zero}, \ctype{hmc\_complex\_i} are available. All according complex operations are defined in \file{operations.h}.
\subsubsection{Spinor types}
Spinor types have only been defined in the C++ part so far:
\begin{verbatim}
typedef hmc_complex hmc_full_spinor [NSPIN*NC];
typedef hmc_complex hmc_full_spinor_field [NSPIN*NC][VOLSPACE][NTIME];
\end{verbatim}
All according operations are defined in \file{operations.h}.
\subsubsection{Gaugefield and matrix types}
There is \ctype{hmc\_su3matrix} which is used as fundamental data type for SU(3) matrices. Depending on the switch \define{\_RECONSRUCT\_TWELVE\_} it is an array of length \verb+NC*(NC-1)+ or a field with \verb+[NC][NC]+ components.
\ctype{hmc\_staplematrix} is an array of length \verb+NC*NC+ (for reconstruct twelve) or simply a new name for \ctype{hmc\_su3matrix} if all 18 components are stored anyways. The gaugefield ist stored as
\begin{verbatim}
#ifdef _RECONSTRUCT_TWELVE_
typedef hmc_complex hmc_gaugefield [NC*(NC-1)][NDIM][VOLSPACE][NTIME];
#else
typedef hmc_complex hmc_gaugefield [NC][NC][NDIM][VOLSPACE][NTIME];
#endif
\end{verbatim}
All according operations are defined in \file{operations.h}.

On device, the typedefs are as follows:
\begin{verbatim}
typedef hmc_complex hmc_ocl_su3matrix;
typedef hmc_complex hmc_ocl_staplematrix;
typedef hmc_float hmc_ocl_gaugefield;
\end{verbatim}
Thus, the necessary array length has to be allocated each time. Operations are given in \file{operations\_kernels.cl}.

\subsection{Operations}
Local and global operations on the types defined in \file{types.h} are contained in \file{operations.h} and \file{operations.cpp}. There are operations on complex types (conjugation, multiplication,\ldots), matrix types (adjoin, trace, determinant,\ldots), spinor types, and gaugefield types.

Besides arithmetic operations there are gaugefield functions to initialise cold and hot start as well as start from source. Furthermore interfaces to get and put SU(3) matrix elements of a gaugefield are provided and functions that copy a gaugefield from hmc type to ocl type.

\subsection{Gaugeobservables}
\file{gaugeobservables.h} and \file{gaugeobservables.cpp} provide
\begin{verbatim}
void print_gaugeobservables(hmc_gaugefield* field);

hmc_float plaquette(hmc_gaugefield * field, hmc_float* tplaq, hmc_float* splaq);
hmc_float plaquette(hmc_gaugefield * field);
hmc_complex polyakov(hmc_gaugefield * field);
hmc_complex spatial_polyakov(hmc_gaugefield * field, int dir);
hmc_complex polyakov_x(hmc_gaugefield * field);
hmc_complex polyakov_y(hmc_gaugefield * field);
hmc_complex polyakov_z(hmc_gaugefield * field);
\end{verbatim}
The print function prints plaquette, temporal and spatial plaquette, real and imaginary part of Polyakov loop to standard out. The spatial polyakov loop functions give consistent results but have not been checked against independent results. The same is true for the spatial and timelike plaquettes.

\file{gaugefieldoperations.h} and \file{gaugefieldoperations.cpp} provide
\begin{verbatim}
void print_info_source(sourcefileparameters* params);

hmc_error init_gaugefield(hmc_gaugefield* gaugefield, inputparameters* parameters, usetimer* timer);
\end{verbatim}
The print function prints xml-info from an input source file to standard out. \function{init\_gaugefield} initialises a (previously allocated) gaugefield according to the start condition. Note the use of timer which needs to be improved.


\subsection{Usetimer}
\file{usetimer.h} defines an ugly wrapper around a timer class defined in \file{timer.h}. We need to improve this\ldots

\subsection{Geometry}
\file{geometry.h} and \file{geometry.cpp} provide a mapping from from $(x,y,z)$ to one int nspace. The naming scheme is as follows:\\
direction 1 = x; direction 2 = y; direction 3 = z; direction 0 = t\\
Important: $(x,y,z)$ are always used together and it should be possible to use a different number of space dimensions (defined in \file{globals.h}). $t$ is always apart.
\begin{verbatim}
//switch between (x,y,z) <-> nspace=0,...,VOLSPACE-1
int get_nspace(int* coord);
int get_spacecoord(int nspace, int dir);

int get_neighbor(int nspace, int dir);
int get_lower_neighbor(int nspace, int dir);
\end{verbatim}

\subsection{Update}
{\bf information on heatbath update should be inserted by Christopher}

\subsection{Opencl}
\file{opencl.h} and \file{opencl.cpp} provide a class \function{opencl} which can be used to initialise an OpenCL device and call the heatbath kernel. The kernels are in \file{\_kernels.cl} files (e.\,g.\ \file{operations\_kernels.cl}). There is a testing kernel which can be called with the according test member function of the OpenCL class. 

The kernel files that should be read in to build the OpenCL \function{clprogram} are listed in the \ctype{string} vector \function{cl\_kernels\_file}. When the programme is built, the complete source code is also written to the file \file{cl\_kernelsource.cl} which is meant to allow for debugging.

\subsection{Testing}
\file{testing.h} and \file{testing.cpp} provide a playground for testing. Just have a look\ldots


\section{Hybrid strategy}
\section{hybrid strategy}

\section{Some readings}
\subsection{ILDG}
Here one can find infos about ILDG: \url{http://ildg.sasr.edu.au/Plone}
\subsection{OpenCL}
To get the OpenCL specifications visit \url{http://www.khronos.org/opencl}.
\subsection{Other GPU-HMCs}
\begin{itemize}
\item Bonati, Cossu, D'Elia, Di Giacomo, Staggered fermions simulations on GPUs~\cite{Bonati:2010qu}
\item {\bf Christopher and Lars: complete this list with all references we know\ldots}
\end{itemize}


\bibliography{literature}


\end{document}