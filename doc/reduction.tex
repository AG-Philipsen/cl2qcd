% Copyright (c) 2015,2018 Alessandro Sciarra
%
% This file is part of CL2QCD.
%
% CL2QCD is free software: you can redistribute it and/or modify
% it under the terms of the GNU General Public License as published by
% the Free Software Foundation, either version 3 of the License, or
% (at your option) any later version.
%
% CL2QCD is distributed in the hope that it will be useful,
% but WITHOUT ANY WARRANTY; without even the implied warranty of
% MERCHANTABILITY or FITNESS FOR A PARTICULAR PURPOSE. See the
% GNU General Public License for more details.
%
% You should have received a copy of the GNU General Public License
% along with CL2QCD. If not, see <http://www.gnu.org/licenses/>.

\documentclass[a4paper,11pt]{article}
\usepackage[english]{babel}
\usepackage[applemac]{inputenc}
\usepackage[T1]{fontenc}
\usepackage{slashed}
\usepackage{amsmath,amssymb,mathabx}
\usepackage{geometry}
\usepackage{xcolor,array}
\usepackage{fancyvrb}       %Personalizzazione ambiente Verbatim

\geometry{textwidth=16cm,textheight=23cm,heightrounded,headsep=12pt,vmarginratio=1:1}

%Commands
\newcommand{\row}[1]{line~\textcolor{red}{#1}}
\newcommand{\rows}[2]{lines~\textcolor{red}{#1}$\div$\textcolor{red}{#2}}
\renewcommand{\theFancyVerbLine}{\textcolor{red}{\footnotesize\arabic{FancyVerbLine}}}
\newcommand{\cut}{\texttt{cut1}}
\newcommand{\ccut}{\texttt{cut2}}
\newcommand{\idx}{\texttt{idx}}
\newcolumntype{P}{>{\rule[-0.3cm]{0pt}{0.8cm}\centering$}p{4mm}<{$}}
\newcolumntype{Q}{>{\rule[-0.4cm]{0pt}{1cm}\centering$}p{9mm}<{$}}
\newcolumntype{R}{>{\rule[-0.4cm]{0pt}{1cm}\centering$}p{24mm}<{$}}
\title{\vspace*{-1.5cm}\huge\textbf {On reduction}\vspace{-0.5cm}}
\author{\small \texttt{of} \emph{Alessandro Sciarra}}
\date{}

\begin{document}

\maketitle

\noindent Let us start recalling the code:
\begin{Verbatim}[gobble=0,numbers=left,numbersep=12pt,stepnumber=1,fontsize=\small]
  //reduction
  if(local_size == 1) {
    ((out))[group_id].re += tmp_pol.re;
    ((out))[group_id].im += tmp_pol.im;
  }else{
    //wait for all threads to end calculations
    barrier(CLK_LOCAL_MEM_FENCE);

   //!!CP: this should be checked by someone else than me
   //perform reduction
   out_loc[idx].re = tmp_pol.re;
   out_loc[idx].im = tmp_pol.im;
   int cut1;
   int cut2 = local_size;
   for(cut1 = local_size / 2; cut1 > 0; cut1 /= 2) {
     for(int i = idx + cut1; i < cut2; i += cut1) {
       ((out_loc)[idx]).re +=  ((out_loc)[i]).re;
       ((out_loc)[idx]).im +=  ((out_loc)[i]).im;
     }
     //!!CP: is this dangerous inside a for-loop?
     barrier(CLK_LOCAL_MEM_FENCE);
     cut2 = cut1;
   }
   if(idx == 0) {
     out[group_id].re = out_loc[0].re;
     out[group_id].im = out_loc[0].im;
   }
 }
\end{Verbatim}
that is nothing but a part of the \texttt{gaugeobservables\_polyakov.cl} file. In order to check if it is correct or not,
a good strategy is to follow it through an example. Actually, we will consider \emph{two} examples, the first with
\texttt{local\_size}$=2^{n}$ and the second with \texttt{local\_size}$\neq2^{n}$.

\subsection*{First example: \texttt{local\_size}$=8$ }

Basically, what we have to do is to check how the components of the vector \texttt{out\_loc} are
summed (and replaced). So, to make the example easier to be understood, we use the following representation of such
a vector:

\begin{center}
\begin{tabular}{!{\vrule width 1.5pt}P*{7}{|P}!{\vrule width 1.5pt}}
\noalign{\hrule height 1.5pt}
0 & 1 & 2 & 3 & 4 & 5 & 6 & 7 \tabularnewline
\noalign{\hrule height 1.5pt}
\end{tabular}
\end{center}

\noindent where in each component we put the index of that cell instead of its content.
We will also indicate the iterations of the outer \texttt{for} loop (\row{15}) with the symbol $\bullet$,
while the inner \texttt{for} loop (\row{16}) iterations will be addressed as $\circ$. Since \texttt{local\_size}$=8$,
the reduction starts with \texttt{cut2}$=8$.
\newpage
\begin{itemize}
%%%
  \item \cut$=\frac{1}{2}\,$\ccut$=4$ that is bigger than 0
  \begin{itemize}
  \item[$\circ$] \idx{} can be $0,1,2,3$ so that \idx$+$\cut$=4,5,6,7$.

\begin{center}
\begin{tabular}{!{\vrule width 1.5pt}Q*{3}{|Q}*{4}{|P}!{\vrule width 1.5pt}}
\noalign{\hrule height 1.5pt}
0+4 & 1+5 & 2+6 & 3+7 & 4 & 5 & 6 & 7 \tabularnewline
\noalign{\hrule height 1.5pt}
\end{tabular}
\end{center}

      \ccut$\;\to\;$\ccut$=$\cut$=4$
  \end{itemize}
%%%
\medskip
%%%
  \item \cut$=\frac{1}{2}\,$\ccut$=2$ that is bigger than 0
  \begin{itemize}
  \item[$\circ$] \idx{} can be $0,1$ so that \idx$+$\cut$=2,3$.

\begin{center}
\begin{tabular}{!{\vrule width 1.5pt}R|R*{2}{|Q}*{4}{|P}!{\vrule width 1.5pt}}
\noalign{\hrule height 1.5pt}
0+4+2+6 & 1+5+3+7 & 2+6 & 3+7 & 4 & 5 & 6 & 7 \tabularnewline
\noalign{\hrule height 1.5pt}
\end{tabular}
\end{center}

      \ccut$\;\to\;$\ccut$=$\cut$=2$
  \end{itemize}
%%%
\medskip
%%%
  \item \cut$=\frac{1}{2}\,$\ccut$=1$ that is bigger than 0
  \begin{itemize}
  \item[$\circ$] \idx{} can be only $0$ so that \idx$+$\cut$=1$.

\begin{center}
\begin{tabular}{!{\vrule width 1.5pt}R|R*{2}{|Q}*{4}{|P}!{\vrule width 1.5pt}}
\noalign{\hrule height 1.5pt}
\sum_{i=0}^{7} i & 1+5+3+7 & 2+6 & 3+7 & 4 & 5 & 6 & 7 \tabularnewline
\noalign{\hrule height 1.5pt}
\end{tabular}
\end{center}

      \ccut$\;\to\;$\ccut$=$\cut$=1$
  \end{itemize}
%%%
\medskip
%%%
  \item \cut$=\frac{1}{2}\,$\ccut$=0$ that is \emph{not} bigger than 0 $\Rightarrow$ \verb"break".
\end{itemize}

\noindent In this example all is fine and there is no race-condition. Obviously, one must be sure that
the barrier call into the \texttt{for} loop (\row{21}) is safe.

\subsection*{Second example: \texttt{local\_size}$=6$ }

Let us now turn to  the other example, with \texttt{local\_size}$\neq2^{n}$: we will see in a moment
that the reduction is no more deterministic. This time, the \texttt{out\_loc} vector will be

\begin{center}
\begin{tabular}{!{\vrule width 1.5pt}P*{5}{|P}!{\vrule width 1.5pt}}
\noalign{\hrule height 1.5pt}
0 & 1 & 2 & 3 & 4 & 5 \tabularnewline
\noalign{\hrule height 1.5pt}
\end{tabular}
\end{center}

\noindent and, since \texttt{local\_size}$=6$, the loops start with \texttt{cut2}$=6$.
\begin{itemize}
%%%
  \item \cut$=\frac{1}{2}\,$\ccut$=3$ that is bigger than 0
  \begin{itemize}
  \item[$\circ$] \idx{} can be $0,1,2$ so that \idx$+$\cut$=3,4,5$.

\begin{center}
\begin{tabular}{!{\vrule width 1.5pt}Q*{2}{|Q}*{3}{|P}!{\vrule width 1.5pt}}
\noalign{\hrule height 1.5pt}
0+3 & 1+4 & 2+5 & 3 & 4 & 5 \tabularnewline
\noalign{\hrule height 1.5pt}
\end{tabular}
\end{center}

      \ccut$\;\to\;$\ccut$=$\cut$=3$
  \end{itemize}
%%%
\medskip
%%%
  \item \cut$=\frac{1}{2}\,$\ccut$=1$ (the result is the cast \texttt{(int)(1.5)}) that is bigger than 0
  \begin{itemize}
  \item[$\circ$] \idx{} can be $0,1$ so that \idx$+$\cut$=1,2$.

\begin{center}
\color{blue}
\begin{tabular}{!{\vrule width 1.5pt}R|R|Q*{3}{|P}!{\vrule width 1.5pt}}
\noalign{\hrule height 1.5pt}
\sum_{i=0}^{5} i  & 1+4+2+5 & 2+5 & 3 & 4 & 5  \tabularnewline
\noalign{\hrule height 1.5pt}
\end{tabular}

\medskip

\color{red}
\begin{tabular}{!{\vrule width 1.5pt}R|R|Q*{3}{|P}!{\vrule width 1.5pt}}
\noalign{\hrule height 1.5pt}
0+3+1+4 & 1+4+2+5 & 2+5 & 3 & 4 & 5 \tabularnewline
\noalign{\hrule height 1.5pt}
\end{tabular}
\end{center}

      \ccut$\;\to\;$\ccut$=$\cut$=1$
  \end{itemize}
%%%
\medskip
%%%
  \item \cut$=\frac{1}{2}\,$\ccut$=0$ that is \emph{not} bigger than 0 $\Rightarrow$ \verb"break".
\end{itemize}

\noindent In the second iteration of the outer \texttt{for} loop, two scenarios were drawn:
\begin{itemize}
  \item[$\star$] \textcolor{blue}{The third cell is added to the second \emph{\underline{before}} that the second is added to the first;}
  \item[$\star$] \textcolor{red}{The third cell is added to the second \emph{\underline{after}} that the second is added to the first.}
\end{itemize}
It is clear that in the first case we get the correct result and that in the second we do not. So we can conclude that
the reduction performed as shown in \rows{15}{23} gives the exact result \emph{if and only if} \texttt{local\_size}$=2^{n}$;
for sure, in such a case, there will be no race-condition and this algorithm is still reliable when the exponent $n$ grows up.










\end{document}
