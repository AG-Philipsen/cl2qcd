% Copyright (c) 2016,2018 Alessandro Sciarra
%
% This file is part of CL2QCD.
%
% CL2QCD is free software: you can redistribute it and/or modify
% it under the terms of the GNU General Public License as published by
% the Free Software Foundation, either version 3 of the License, or
% (at your option) any later version.
%
% CL2QCD is distributed in the hope that it will be useful,
% but WITHOUT ANY WARRANTY; without even the implied warranty of
% MERCHANTABILITY or FITNESS FOR A PARTICULAR PURPOSE. See the
% GNU General Public License for more details.
%
% You should have received a copy of the GNU General Public License
% along with CL2QCD. If not, see <http://www.gnu.org/licenses/>.

\documentclass[a4paper,10pt]{article}
\usepackage[utf8]{inputenc}
\usepackage[english]{babel}
\usepackage[dvipsnames,usenames]{xcolor}
\usepackage{geometry}
\usepackage{indentfirst}
\usepackage{amsmath,amssymb,bbm,mathabx}
\usepackage{slashed}
\usepackage{microtype}
\usepackage{braket}
\usepackage{simplewick}
\usepackage{graphicx}
\usepackage{cleveref}
\usepackage{cancel}
\usepackage{nameref}

%geometry
\geometry{textheight=23cm,textwidth=17cm,hmarginratio=1:1,vmarginratio=1:1}
%cleveref
\crefname{equation}{Eq.}{Eqs.}

%Commands
\newcommand{\Eq}[1]{Eq.~\eqref{eq:#1}}
\newcommand{\Eqs}[1]{Eqs.~\eqref{eq:#1}}
\newcommand{\M}[1]{\color{OliveGreen}#1\color{black}}
\renewcommand*\thesubsection{\arabic{subsection}}

%opening
\title{\vspace*{-2cm}\huge{\textbf{On boundary conditions in simulations}}\vspace{-0.2cm}}
\author{\texttt{of} \emph{Alessandro Sciarra}}
\date{}

\begin{document}

\maketitle

The way to impose some particular boundary conditions (BC) in a simulation is not unique. Let us suppose
that we want the fermionic field satisfy the following property:
\begin{subequations}\label{eq:1}
\begin{equation}\label{eq:1a}
 \chi(n+N_\mu\,\hat{e}_\mu)=e^{\imath\theta_\mu}\chi(n) \;,
\end{equation}
where $n\equiv(n_x,n_y,n_z,n_t)$, $N_\mu$ is the extension of the lattice in the $\mu$--direction and $\theta_\mu$ is a
global parameter\footnote{The BC on the gauge field are assumed to be periodic.}. Observe that \Eqs{1} implies
\begin{equation}\label{eq:1b}
 \chi(n-N_\mu\,\hat{e}_\mu)=e^{-\imath\theta_\mu}\chi(n) \;,
\end{equation}
\end{subequations}
sending $n\to n-N_\mu\,\hat{e}_\mu$. We will focus only on the staggered formulation
and, in particular, we will recall the two main possibilities to impose the BC of \Eqs{1}.
It is quite easy to apply all the following arguments to different formulations (e.g. Wilson
fermions), but we prefer not to do that, leaving this to the interested Reader.

\noindent Just to fix the notation, we will refer to the standard staggered action as
\begin{equation}\label{eq:2}
  S=\frac{1}{2}\sum_{n,\mu}\bar\chi(n)\eta_\mu(n)\,\Bigl[U_\mu(n)^{\phantom{\dag}}\chi(n+\hat\mu)
					  -U^\dag_\mu(n-\hat\mu)\chi(n-\hat\mu)\Bigr]+M_0\sum_n \bar\chi(n)\chi(n)\;,
\end{equation}
where
\begin{equation*}
 \left\{
   \begin{aligned}
     \eta_1(n)&=1 \\
     \eta_\mu(n)&=(-1)^{\sum_{\nu<\mu}n_\nu} \quad\mbox{if}\quad \mu\neq1
   \end{aligned}
  \right.
\end{equation*}
and the color index has been omitted.

\subsection{Distributing the BC on the whole lattice}\label{Str1}

One very easy possibility to impose \Eqs{1} is to make a unitary abelian transformation on the fields that
we want to satisfy such BC\footnote{See \texttt{arXiv:hep-lat/0405002v2}.}.
In particular, let us consider the following transformation\footnote{Remark that
$T(n\pm N_\mu\hat{e}_\mu)=e^{\mp\imath\theta_\mu}\:T(n)\:$, where
$T(n)=\exp\bigl(-\imath\sum_\mu\frac{\theta_\mu\:n_\mu}{N_\mu}\bigr)$.}:
\begin{subequations}\label{eq:3}
\begin{align}
 \chi'(n) &=e^{-\imath\sum_\mu\frac{\theta_\mu\:n_\mu}{N_\mu}}\:\chi(n) \\
 \bar\chi'(n) &=e^{+\imath\sum_\mu\frac{\theta_\mu\:n_\mu}{N_\mu}}\:\bar\chi(n)\;;
\end{align}
\end{subequations}
it is easy to show that, if the $\chi'$ field has \emph{periodic} BC in every direction, then the $\chi$
field satisfy \Eqs{1}.

The transformation above in the standard staggered action leads to:
\begin{equation}\label{eq:4}
  S=\frac{1}{2}\sum_{n,\mu}\bar\chi'(n)\eta_\mu(n)\,\Bigl[e^{\imath\frac{\theta_\mu}{N_\mu}}\:
	U_\mu(n)^{\phantom{\dag}}\chi'(n+\hat\mu)-e^{-\imath\frac{\theta_\mu}{N_\mu}}\:
	U^\dag_\mu(n-\hat\mu)\chi'(n-\hat\mu)\Bigr]+M_0\sum_n \bar\chi'(n)\chi'(n)\;.
\end{equation}
From \Eq{4} we can state that \emph{during a simulation} one can impose the desired BC by using a
fermionic field that has \emph{periodic} BC, but that obeys a modified Dirac equation.
In other words, it is sufficient to multiply \underline{each} link by the proper phase.
For future convenience, let us indicate the modified Dirac operator as
\begin{equation*}
 M^\theta_{n,m} \equiv \frac{1}{2}\sum_{\mu=1}^4\eta_\mu(n)\,\Bigl[
                 e^{\imath\frac{\theta_\mu}{N_\mu}}\:U_\mu(n)^{\phantom{\dag}}\delta_{n+\hat\mu,m}
                -e^{-\imath\frac{\theta_\mu}{N_\mu}}\:U^\dag_\mu(n-\hat\mu)\delta_{n-\hat\mu,m}\Bigr]+M_0\delta_{n,m}\;.
\end{equation*}

\textcolor{red}{It is worth remarking that \Eq{4} is nothing more than a rewriting of the
action, and no particular simmetry of the theory has been used.} It is simply easier by
the programming point of view to deal with periodic fermionic fields (in every direction),
because then it has not to be checked in the code whether the boundary of the lattice
is beeing crossed\footnote{In principle, one could directly implement \cref{eq:1a,eq:1b} using
some \emph{if}--statements evaluating the coordinates of the forward and backward next-neighbor.
Nevertheless this is particularly time consuming because one should continuously check where
he is on the lattice.}.

% For future convenience, let us introduce some new notation to rewrite \Eqs{3}. Defining
% \[
%  T_\mu(n) \equiv e^{\imath \frac{n_\mu\,\theta_\mu}{N_\mu}}\;,
% \]
% we have that
% \[
%  e^{+\imath\sum_\mu\frac{\theta_\mu\:n_\mu}{N_\mu}}=\prod_{\mu=1}^4 T_\mu(n) \equiv T(n)\;.
% \]
% Obviously, from \Eqs{3} we can write
% \[
%  \chi(n) =T(n)\:\chi(n) \qquad\Leftrightarrow\qquad
%  \bar\chi(n) =T^\dag(n)\:\bar\chi(n)\;,
% \]
% and it is important to remark that
% \[
%  T(n\pm N_\mu\hat{e}_\mu)=e^{\pm\imath\theta_\mu}\:T(n)\;.
% \]


\subsection{Leaving the BC at the end of the lattice}\label{Str2}

Another possibility to impose the desired BC is to leave them at some point on the lattice.
This statement could sound completely the opposite of what said above, but there is a trick
to easily do that. Basically, the staggered phases $\eta_\mu(n)$ can be included in links
and forgotten during a simulation. Before doing that, one can multiply one staggered phase in
each direction by a proper factor, imposing in this way \Eqs{1}. Let us see now in more
details how this preocedure works.

First of all let us remark that, without loss of generality, we can choose to impose \Eqs{1}
at the end of the lattice in each direction, namely
\begin{equation}\label{eq:5}
  \begin{aligned}
    \chi(N_\mu\,\hat{e}_\mu)&=e^{\imath\theta_\mu}\chi(\underline{0}) \\
    \chi(-1\,\hat{e}_\mu)&=e^{-\imath\theta_\mu}\chi((N_\mu-1)\,\hat{e}_\mu) \;,
  \end{aligned}
\end{equation}
with $n=(0,0,0,0)\equiv\underline{0}$. Observe that the other three arguments of the $\chi$-field
in the equations above are to be read as four vectors $(n_x,n_y,n_z,n_t)$ with $\forall n_\nu$ for $\nu\neq\mu$
(and $n_\nu$ at the end of the lattice for $\nu=\mu$). So, one can think that the fermionic field takes a
phase \emph{only} between the last and the first site in each direction. Now, to gather some ideas
about the general argument, let us consider only non-periodic BC in the time direction (the generalization
to the other case is trivial and it will be made later on). Let us indicate the site coordinates as
$n\equiv(\vec{n},t)$: \Eq{1a} and \Eq{1b} become
\begin{subequations}\label{eq:6}
  \begin{equation}\label{eq:6a}
    \begin{aligned}
      \chi(\vec{n},\:N_t) &=e^{\imath\theta_t}\:\chi(\vec{n},\:0) \\
      \chi(\vec{n},\:-1)  &=e^{-\imath\theta_t}\:\chi(\vec{n},\:N_t-1) \;.
    \end{aligned}
  \end{equation}
for the temporal direction and
  \begin{equation}\label{eq:6b}
    \begin{aligned}
      \chi(N_i\,\hat{e}_i,\:t) &=\chi(\vec{0},\:t) \\
      \chi(-1\,\hat{e}_i,\:t)  &=\chi((N_i-1)\,\hat{e}_i,\:t) \;.
    \end{aligned}
  \end{equation}
\end{subequations}
for spatial directions. Again, the other three arguments of the $\chi$-field
in the equations above are to be read as vectors $(n_x,n_y,n_z)$ with $\forall n_j=0$ for $j\neq i$
(and $n_j$ at the end of the lattice for $j=i$). Defining $M_{n,m}$ so that
\[
 S=\sum_{n,m}\bar\chi(n)\:M_{n,m}\:\chi(m) \;,
\]
it is easy to see that
\begin{equation}\label{eq:7}
 \chi(n)=\sum_m M_{n,m}\:\chi(m)=\frac{1}{2}\sum_\mu \eta_\mu(n)\,\Bigl[U_\mu(n)^{\phantom{\dag}}\chi(n+\hat\mu)
					  -U^\dag_\mu(n-\hat\mu)\chi(n-\hat\mu)\Bigr]+M_0 \chi(n)\;.
\end{equation}
Obviously, there is an expression like \Eq{7} for each site of the lattice. Nevertheless, the
temporal phase $\theta_t$ will appear only if $n=(\vec{n},\:0)$ and $n=(\vec{n},\:N_t-1)$; in particular:
\begin{subequations}\label{eq:8}
\begin{align}
 \chi(\vec{n},\:0)&=\frac{1}{2}\:\eta_t(\vec{n},\:0)\Bigl[U^{\phantom{\dag}}_t(\vec{n},\:0)\:\chi(\vec{n},\:1)
					-U^\dag_t(\vec{n},\:-1)\:\chi(\vec{n},\:-1)\Bigr]+(\dots)= \notag \\[0.5ex]
	&=\frac{1}{2}\:\eta_t(\vec{n},\:0)\Bigl[U^{\phantom{\dag}}_t(\vec{n},\:0)\:\chi(\vec{n},\:1)
		-e^{-\imath\theta_t}U^\dag_t(\vec{n},\:N_t-1)\:\chi(\vec{n},\:N_t-1)\Bigr]+(\dots)= \notag \\[0.5ex]
	&=\frac{1}{2}\:\Bigl[\eta_t(\vec{n},\:0)U^{\phantom{\dag}}_t(\vec{n},\:0)\:\chi(\vec{n},\:1)
		-e^{-\imath\theta_t}\eta_t(\vec{n},\:\textcolor{blue}{N_t-1})U^\dag_t(\vec{n},\:N_t-1)
		\:\chi(\vec{n},\:N_t-1)\Bigr]+(\dots) =\notag\\[0.5ex]
	&=\frac{1}{2}\:\Bigl[\eta_t(\vec{n},\:0)U^{\phantom{\dag}}_t(\vec{n},\:0)\:\chi(\vec{n},\:1)
		-\Bigl(e^{\imath\theta_t}\eta_t(\vec{n},\:N_t-1)\Bigr)^{\star}U^\dag_t(\vec{n},\:N_t-1)
		\:\chi(\vec{n},\:N_t-1)\Bigr]+(\dots) \displaybreak[3]\\[1.5ex]
 \chi(\vec{n},\:N_t-1)&=\frac{1}{2}\:\eta_t(\vec{n},\:N_t-1)\Bigl[U^{\phantom{\dag}}_t(\vec{n},\:N_t-1)\:\chi(\vec{n},\:N_t)
					-U^\dag_t(\vec{n},\:N_t-2)\:\chi(\vec{n},\:N_t-2)\Bigr]+(\dots)= \notag \\[0.5ex]
	&=\frac{1}{2}\:\eta_t(\vec{n},\:N_t-1)\Bigl[e^{\imath\theta_t}U^{\phantom{\dag}}_t(\vec{n},\:N_t-1)
		\:\chi(\vec{n},\:0)-U^\dag_t(\vec{n},\:N_t-2)\:\chi(\vec{n},\:N_t-2)\Bigr]+(\dots)=  \\[0.5ex]
	&=\frac{1}{2}\:\Bigl[e^{\imath\theta_t}\eta_t(\vec{n},N_t-1)U^{\phantom{\dag}}_t(\vec{n},\:N_t-1)\:\chi(\vec{n},\:0)
		-\eta_t(\vec{n},\:\textcolor{blue}{N_t-2})U^\dag_t(\vec{n},\:N_t-2)\:\chi(\vec{n},\:N_t-2)
		\Bigr]+(\dots)\;;\notag
\end{align}
\end{subequations}
where the $(\dots)$ include the remaining terms. In both first steps, \Eqs{6a} were used, while in both last steps
we took into account that the staggered phase $\eta_\mu(n)$ does not depend on $n_\mu$. From \Eqs{8} it is easy
to read how to impose \emph{during a simulation} the BC~\eqref{eq:5}: on condition that each staggered phase
appears \emph{exactly} next to his link, it is sufficient to multiply the staggered phases
$\eta_t(\vec{n},N_t-1)$ by $e^{\imath\theta_t}$ and to take the complex conjugate whenever they
appear next to the hermitean conjugate of a link. It is then clear why staggered phases are often
included in links (remark that, if this is the case, the complex conjugate is automatically taken,
thanks to the $\dag$-operator). \textcolor{red}{It is worth remarking that once we modify staggered
phases, then we have to make the fermionic field satisfy \emph{periodic} BC in \emph{every} direction,
because otherwise the BC phase would appear more than only once.}

\smallskip

To impose the general condition reported in \Eqs{1}, it will be necessary to multiply the
staggered phases in direction $\mu$ by $e^{\imath\theta_\mu}$ at the end of the lattice
in such a direction (and, again, make the phases appear next to their links and take
the complex conjugate properly). Here, again, we can define a modified Dirac operator as
\begin{equation*}
 \tilde{M}_{n,m}\equiv\frac{1}{2}\sum_{\mu=1}^4\eta_\mu(n)\,\Bigl[
                      \tilde{U}'_\mu(n)^{\phantom{\dag}}\delta_{n+\hat\mu,m}-
                      \tilde{U}'^{\:\dag}_\mu(n-\hat\mu)\delta_{n-\hat\mu,m}\Bigr]+M_0\delta_{n,m}\;,
\end{equation*}
where
\begin{alignat*}{3}
 \bullet& \quad\tilde{U}'_\mu(n)&&=\begin{cases} U_\mu(n) &\text{if }n\neq N_\mu-1 \\
                                 e^{\imath\theta_\mu} U_\mu(n) &\text{if }n=N_\mu-1 \end{cases} \qquad&&,\\[1ex]
 \bullet& \quad\tilde{U}'^{\:\dag}_\mu(n)&&=\begin{cases} U_\mu(n) &\text{if }n\neq 0 \\
                                 e^{-\imath\theta_\mu} U_\mu(n) &\text{if }n=0 \end{cases}&&.
\end{alignat*}


\section*{Are the two strategies connected?}

So far we explained how to behave in a simulation to impose the desired BC, but we did not
really justified why we are allowed to procede in such a way. Furthermore anyone will be
wondering whether the two possibilities above depicted are or not equivalent and, in case, why.
To answer this question, let us go through the second procedure once again, this time in a
more formal way. At the end it will be clear which is the connection between the two strategies
and how they are linked.

\medskip

Let us start again from the action of the theory,
\begin{equation*}
  S=\frac{1}{2}\sum_{n,\mu}\bar\chi(n)\eta_\mu(n)\,\Bigl[U_\mu(n)^{\phantom{\dag}}\chi(n+\hat\mu)
					  -U^\dag_\mu(n-\hat\mu)\chi(n-\hat\mu)\Bigr]+M_0\sum_n \bar\chi(n)\chi(n)\;,
\end{equation*}
where the fermionic field $\chi$ satisfies the BC~\eqref{eq:1}. If we make the transformation done
in §\ref{Str1} we pass to fermionic fields with periodic BC in all directions and the action becomes
\begin{equation*}
  S=\frac{1}{2}\sum_{n,\mu}\bar\chi'(n)\eta_\mu(n)\,\Bigl[e^{\imath\frac{\theta_\mu}{N_\mu}}\:
	U_\mu(n)^{\phantom{\dag}}\chi'(n+\hat\mu)-e^{-\imath\frac{\theta_\mu}{N_\mu}}\:
	U^\dag_\mu(n-\hat\mu)\chi'(n-\hat\mu)\Bigr]+M_0\sum_n \bar\chi'(n)\chi'(n)\;,
\end{equation*}
exactly the same as in \Eq{4}. Then we perform another transformation, this time on links, that reads
\begin{equation}\label{eq:9}
 U'_\mu(n)=G(n)\:U_\mu(n)\:G^{-1}(n+\hat\mu) \;,
\end{equation}
where
\begin{equation*}
 G(n)=e^{-\imath\sum_\mu\frac{\theta_\mu\:n_\mu}{N_\mu}}\qquad \mbox{with}\qquad  G(n\pm N_\mu\hat{e}_\mu)=G(n)\;.
\end{equation*}
Putting \Eq{9} in \Eq{10}, we have
\begin{align*}
 S=\frac{1}{2}\sum_{n,\mu}\bar\chi'(n)\eta_\mu(n)\,\Bigl[&e^{\imath\frac{\theta_\mu}{N_\mu}}\:
	G^{-1}(n)\:U'_\mu(n)^{\phantom{\dag}}G(n+\hat\mu)\:\chi'(n+\hat\mu)+\notag\\
	-&e^{-\imath\frac{\theta_\mu}{N_\mu}}\:
	G^{-1}(n)\:U'^{\:\dag}_\mu(n-\hat\mu)G(n-\hat\mu)\chi'(n-\hat\mu)\Bigr]+M_0\sum_n \bar\chi'(n)\chi'(n)\;.
\end{align*}
Now, with a bit of algebra, we have the following cases
\begin{alignat*}{2}
  \bullet& \quad e^{\imath\frac{\theta_\mu}{N_\mu}}\:G(n)^{-1}\:G(n+\hat\mu)&&=\left\{
        \begin{aligned}
         1 &\quad\mbox{if}\quad n_\mu \neq N_\mu-1 \\
         e^{\imath\theta_\mu} &\quad\mbox{if}\quad n_\mu = N_\mu-1
        \end{aligned}
        \right.\\[1ex]
  \bullet& \quad e^{-\imath\frac{\theta_\mu}{N_\mu}}\:G(n)^{-1}\:G(n-\hat\mu)&&=\left\{
        \begin{aligned}
         1 &\quad\mbox{if}\quad n_\mu \neq 0 \\
         e^{-\imath\theta_\mu} &\quad\mbox{if}\quad n_\mu = 0
        \end{aligned}
        \right.
\end{alignat*}
that put in the action give
\begin{equation}\label{eq:10}
 S=\frac{1}{2}\sum_{n,\mu}\bar\chi'(n)\eta_\mu(n)\,
   \Bigl[\tilde{U}'_\mu(n)^{\phantom{\dag}}\:\chi'(n+\hat\mu)-
         \tilde{U}'^{\:\dag}_\mu(n-\hat\mu)\chi'(n-\hat\mu)\Bigr]+M_0\sum_n \bar\chi'(n)\chi'(n)\;,
\end{equation}
where $\tilde{U}'_\mu(n)$ and $\tilde{U}'^{\:\dag}_\mu(n)$ are the same as those defined at the
end of §\ref{Str2}. We can now easily rewrite \Eq{10} making staggered phases complex and moving
them next to (and ideally also inside) the links:
\begin{equation}\label{eq:11}
 S=\frac{1}{2}\sum_{n,\mu}\bar\chi'(n)\,
   \Bigl[\tilde{\eta}_\mu(n)\:U'_\mu(n)^{\phantom{\dag}}\chi'(n+\hat\mu)-
         \tilde{\eta}^\star_\mu(n-\hat\mu)\:U'^{\:\dag}_\mu(n-\hat\mu)\chi'(n-\hat\mu)\Bigr]+M_0\sum_n \bar\chi'(n)\chi'(n)\;,
\end{equation}
with
\[
 \tilde{\eta}_\mu(n)=\begin{cases} \eta_\mu(n) &\text{if }n\neq N_\mu-1 \\
                                 e^{\imath\theta_\mu} \eta_\mu(n) &\text{if }n=N_\mu-1 \end{cases} \qquad
                                 \text{and}\qquad \tilde{\eta}_\mu(-\hat\mu)=\eta_\mu(N_\mu-1)\;.
\]
Once again we obtained the same result of §\ref{Str2}: to impose BC~\eqref{eq:1}, it is
necessary to multiply the staggered phases in direction $\mu$ by $e^{\imath\theta_\mu}$
at the end of the lattice in such a direction (and make the phases appear \emph{exactly}
next to their links and take the complex conjugate properly).

\medskip

We are now ready to compare the two strategies. From the previous detailed analysis it should
be clear that \textcolor{blue}{\emph{to pass from strategy §\ref{Str1} to strategy §\ref{Str2}
one has to transform the link configuration accordingly to \Eq{9}}}. Then, to understand
why this transformation is allowed, one can give the following argument. The partition
function on which the (R)HMC algorithm is based reads
\begin{equation}\label{eq:12}
 \mathcal{Z}=\int DU \:\text{det}M\: e^{-S[U]}\;,
\end{equation}
where det$M$ is the fermionic determinant that arises from the functional integration on fermionic
fields, while $S[U]$ is the gauge part of the action. To gauge transform $\mathcal{Z}$, one only
need to send $U_\mu(n)$ into $G(n)\:U_\mu(n)\:G^{-1}(n+\hat\mu)$, since it is the only present field.
It easily turns out that $\mathcal{Z}$ is invariant under this transformation\footnote{The Haar measure
$DU$ and $S[U]$ are invariant by definition; to prove that det$M$ is invariant one can rewrite it as
\[
 \text{det}M=\exp[Tr(\ln M)]
\]
and then expand the $\ln(M_0\:\mathbbm{1}+D)$ in series. The trace will make
only closed loops on the lattice survive and then each term of the series is explicitly gauge invariant.}.
Then, the two strategies lead to algorithm based on
\[
 \mathcal{Z}^\theta=\int DU \:\text{det}M^\theta\: e^{-S[U]}\qquad \text{from §\ref{Str1}}\;,
\]
and
\[
 \tilde{\mathcal{Z}}=\int DU \:\text{det}\tilde{M}\: e^{-S[U]}\qquad \text{from §\ref{Str2}}\;.
\]
But we showed that
\[
 M^\theta\;\xrightarrow{U'=GUG^{-1}}\;\tilde{M}\qquad\Rightarrow\qquad \mathcal{Z}^\theta=\tilde{\mathcal{Z}}\;,
\]
from which we conclude that both algorithms give the same expectation values of gauge invariant observables.










\color{OliveGreen}
\subsubsection*{Thought:}

It is well known that the action is invariant under local gauge transformations,
then we can do one without changing the physics described by $S$. Recalling that
on the lattice
\begin{subequations}\label{eq:A1}
  \begin{align}
   \chi(n) \,&\to\, \chi'(n)=G(n)\:\chi(n) \label{eq:A1a}\\
   U_\mu(n) \,&\to\, U'_\mu(n)=G(n)\:U_\mu(n)\:G^{-1}(n+\hat\mu) \;, \label{eq:A1b}
  \end{align}
\end{subequations}
under the gauge transformation $G(n)$, let us consider
\begin{equation}\label{eq:A2}
 G(n)=e^{-\imath\sum_\mu\frac{\theta_\mu\:n_\mu}{N_\mu}}\;.
\end{equation}
Of course, \Eqs{A1} do not make $S$ change. Ignoring the mass term, we have
\begin{align}\label{eq:A3}
 S'=\frac{1}{2}\sum_{n,\mu} \eta_\mu(n)\:\bar\chi'(n)\:G(n)\,
                     \Bigl[&G^{-1}(n)\:U'_\mu(n)^{\phantom{\dag}}G(n+\hat\mu)\:G^{-1}(n+\hat\mu)\:\chi'(n+\hat\mu)+\notag\\
                    -&G^{-1}(n)\:U'^{\:\dag}_\mu(n-\hat\mu)G(n-\hat\mu)\:G^{-1}(n-\hat\mu)\:\chi'(n-\hat\mu)\Bigr]=S\;.
\end{align}
In \Eq{A3} we fixed how $G(n)$ behaves wrapping once around the lattice: it does not matter how it does
but this behaviour must be fixed to leave $S$ unchanged. On the contrary, the strategy depicted in the
section \emph{\nameref{Str2}} is equivalent to put \Eqs{A1} into $S$ making $G(n)$ behave when wrapping
around the lattice as the field it is acting on. This means that in \Eq{A1a} it is
\[
 e^{-\imath\sum_\mu\frac{\theta_\mu\:n_\mu}{N_\mu}}
 \qquad\Leftrightarrow\qquad G(n\pm N_\mu\hat{e}_\mu)=e^{\mp\imath\theta_\mu}\:G(n)\;,
\]
while in \Eq{A1b} it is
\[
 e^{-\imath\sum_\mu\frac{\theta_\mu\:(n_\mu)(\text{mod }N_\mu)}{N_\mu}}
 \qquad\Leftrightarrow\qquad G(n\pm N_\mu\hat{e}_\mu)=G(n)\;.
\]













% \subsection*{A very simple test whose result remains unclear...}
%
% As well known, during the development of a code, it is very useful to implement tests
% for its different parts. For the Dirac operator, for example, a good test is to calculate with a
% Reference Code the squarenorm of the field $\chi(n)=\sum_m M_{n,m}\:\chi(m)$, and then
% to repeat such a calculation with the own code, obtaining (hopefully) the same squarenorm.
% At first, one can start with a very very simple set-up, i.e. all links equal to $\mathbbm{1}$,
% all fields $\chi(m)=(1,1,1)$, the mass equal to zero and, finally, some BC.
%
% \smallskip
%
% \noindent Let us focus on the following set-up:
% \begin{itemize}
%  \item $N_\mu=4\quad \forall \mu$;
%  \item $\chi(m)=(1,1,1)\equiv\vec{v}\quad \forall m$;
%  \item $U_t(\vec{0},0)$ random matrix of $SU(3)$ and all other links equal to $\mathbbm{1}_{3\times3}$;
%  \item Periodic BC in spatial directions and \emph{antiperiodic} BC in the temporal direction
%        ($\theta_t=1$ and $\theta_i=0$);
%  \item $M_0=0$.
% \end{itemize}
% In this situation it is possible to calculate the result a priori: let us do such a calculation
% first imposing the BC modifing the staggered phases, and then tranforming the staggered field.
%
% \subsubsection*{First method:}
%
% Calculating the field $\chi(n)$ following the first procedure, i.e. imposing \Eqs{3} and \Eqs{3b},
% it is easy to demonstrate that $\chi(n)$ does not vanish only if
% $n=(\vec{n},\:0)$, $n=(\vec{0},\:1)$ or $n=(\vec{n},\:N_t-1)$.
% In these three cases, indicating with $U$ the only link not equal to $\mathbbm{1}$, we get:
% \begin{subequations}\label{eq:8}
% \begin{alignat}{2}
%  \chi(\vec{0},\:0)&=\frac{1}{2}\:\eta_t(\vec{0},\:0)\;\Bigl[U+\mathbbm{1}\Bigr]\cdot\vec{v}  \\[0.5ex]
%  \chi(\vec{0},\:1)&=\frac{1}{2}\:\eta_t(\vec{0},\:1)\;\Bigl[\mathbbm{1}-U^\dag\Bigr]\cdot\vec{v}  \\[0.5ex]
%  \chi(\vec{n},\:0)&=\frac{1}{2}\:\eta_t(\vec{n},\:0)\;\Bigl[\mathbbm{1}+\mathbbm{1}\Bigr]\cdot\vec{v}
%   &\qquad\quad&\forall\:\vec{n}\neq\vec{0} \\[0.5ex]
%  \chi(\vec{n},\:N_t-1)&=\frac{1}{2}\:\eta_t(\vec{n},\:N_t-1)\;\Bigl[-\mathbbm{1}-\mathbbm{1}\Bigr]\cdot\vec{v}
%  &&\forall\:\vec{n}\;;
% \end{alignat}
% \end{subequations}
% therefore
% \begin{equation*}
%  |\chi|^2=\sum_{\vec{n}\neq\vec{0}}\;\Bigl\{\bigl|\chi(\vec{n},0)\bigr|^2+\bigl|\chi(\vec{n},N_t-1)\bigr|^2\Bigr\}
%           +\bigr|\chi(\vec{0},0)\bigl|^2+\bigr|\chi(\vec{0},1)\bigl|^2+\bigl|\chi(\vec{0},N_t-1)\bigr|^2\;.
% \end{equation*}
% Since
% \[
% \begin{aligned}
%  4\cdot\bigl|\chi(\vec{0},0)\bigr|^2 &= \vec{v}^\dag\cdot\bigl(\mathbbm{1}+U^\dag\bigr)\cdot
% 					\bigl(\mathbbm{1}+U\bigr)\cdot\vec{v}=
% 					\vec{v}^\dag\cdot\bigl(2\:\mathbbm{1}+U^\dag+U\bigr)\cdot\vec{v}\\[0.5ex]
%  4\cdot\bigl|\chi(\vec{0},1)\bigr|^2 &= \vec{v}^\dag\cdot\bigl(\mathbbm{1}-U\bigr)\cdot
% 					\bigl(\mathbbm{1}-U^\dag\bigr)\cdot\vec{v}=
% 					\vec{v}^\dag\cdot\bigl(2\:\mathbbm{1}-U^\dag-U\bigr)\cdot\vec{v}\\[0.5ex]
%  \bigl|\chi(\vec{n},0)\bigr|^2 &= \vec{v}^\dag\cdot\vec{v}\qquad\quad\forall\:\vec{n}\neq\vec{0}\\[0.5ex]
%  \bigl|\chi(\vec{n},N_t-1)\bigr|^2 &= \vec{v}^\dag\cdot\vec{v}\qquad\quad\forall\:\vec{n}\;,
% \end{aligned}
% \]
% then we obtain
% \begin{equation}\label{eq:9}
%  |\chi|^2=\sum_{\vec{n}\neq\vec{0}}\;\bigl( 2\,\vec{v}^\dag\cdot\vec{v}\bigr)
%          + 2\, \vec{v}^\dag\cdot\vec{v}
%          =(4^3-1)\cdot2\cdot 3+2\cdot3=384\;.
% \end{equation}
%
% \subsubsection*{Second method:}
%
% Let us now try to derive again the result of \Eq{9}, following the second procedure, i.e. following
% \Eq{7}. It is worth remarking that fields $\chi$ and $\chi'$ have the same squarenorm, as we can see
% from \Eqs{6}:
% \begin{equation*}
%  |\chi'|^2=\sum_n\:\bigl|\chi'(n)\bigr|^2=\sum_n\:\Bigl|e^{-\imath\sum_\mu\frac{\theta_\mu\:n_\mu}{N_\mu}}\:\chi(n)\Bigr|^2
%          = \sum_n\:\bigl|\chi(n)\bigr|^2=|\chi|^2\;;
% \end{equation*}
% this allows us to work from now on with the field $\chi'$ which we will refer to simply as $\chi$. Furthermore,
% the equation above justifies somehow why we expect the same test result with both procedures.
%
% This time, the field $\chi(n)$ will be non-zero on all the lattice ($\alpha=1/4$):
%
% \begin{subequations}\label{eq:10}
% \begin{align}
%  \chi(\vec{0},\:0)&=\frac{1}{2}\:\eta_t(\vec{n},\:0)\;\Bigl[e^{\imath\alpha\pi}U-
% 					e^{-\imath\alpha\pi}\mathbbm{1}\Bigr]\cdot\vec{v}  \\[0.5ex]
%  \chi(\vec{0},\:1)&=\frac{1}{2}\:\eta_t(\vec{n},\:1)\;\Bigl[e^{\imath\alpha\pi}\mathbbm{1}-
% 					e^{-\imath\alpha\pi}U^\dag\Bigr]\cdot\vec{v}  \\[0.5ex]
%  \chi(\vec{n},\:t)&=\frac{1}{2}\:\eta_t(\vec{n},\:t)\;\Bigl[e^{\imath\alpha\pi}-
% 					e^{-\imath\alpha\pi}\Bigr]\mathbbm{1}\cdot\vec{v} \label{eq:10c}
% \end{align}
% \end{subequations}
% where in \Eq{10c} $n\neq(\vec{0},0)$ and $n\neq(\vec{0},1)$. Hence
% \begin{equation*}
%  |\chi|^2=\bigl|\chi(\vec{0},0)\bigr|^2+\bigr|\chi(\vec{0},1)\bigl|^2+
% 	  \sum_{\substack{n\neq(\vec{0},0)\\n\neq(\vec{0},1)}}\bigl|\chi(\vec{n},t)\bigr|^2\;.
% \end{equation*}
% It is easy to calculate explicitly the last term:
% \begin{equation}\label{eq:11}
%   \sum_{\substack{n\neq(\vec{0},0)\\n\neq(\vec{0},1)}}\bigl|\chi(\vec{n},t)\bigr|^2=
%   (4^4-2)\cdot\sin^2(\alpha\pi)\cdot|\vec{v}|^2=381\;.
% \end{equation}
% Regarding the other two terms we have:
% \[
%   \bigl|\chi(\vec{n},0)\bigr|^2 =\bigl|\chi(\vec{n},1)\bigr|^2 =
%   \frac{1}{4}\;\vec{v}^\dag\cdot\bigl(2\:\mathbbm{1}-e^{-2\imath\alpha\pi}U^\dag-
%   e^{2\imath\alpha\pi}U\bigr)\cdot\vec{v}\;,
% \]
% therefore
% \begin{align}\label{eq:12}
%  \bigl|\chi(\vec{n},0)\bigr|^2+\bigl|\chi(\vec{n},1)\bigr|^2 &=
%  \vec{v}^\dag\cdot\vec{v}-2\:\Re\bigl(e^{2\imath\alpha\pi}U\bigr)=\notag\\[0.75ex]
%  &=3-\vec{v}^\dag\cdot\Bigl[\Re\bigl(e^{2\imath\alpha\pi}\bigr)\,\Re(U)
%                           -\Im\bigl(e^{2\imath\alpha\pi}\bigr)\,\Im(U)\Bigr]\cdot\vec{v}=\notag\\[0.75ex]
%  &=3-\vec{v}^\dag\cdot\Bigl[\cos(2\imath\alpha\pi)\,\Re(U)
%                            -\sin(2\imath\alpha\pi)\,\Im(U)\Bigr]\cdot\vec{v}=\notag\\[0.5ex]
%  &=3+\sum_{i=1}^3\sum_{j=1}^3 \Im\bigl(U_{i,j}\bigr) \;,
% \end{align}
% where we used the explicit value of $\alpha$. Putting \Eq{11} and \Eq{12} together, we get
% \begin{equation}\label{eq:13}
%  |\chi|^2=384+\sum_{i=1}^3\sum_{j=1}^3 \Im\bigl(U_{i,j}\bigr)\;.
% \end{equation}
% Now, if we want \Eq{13} agree with \Eq{9}, then $\sum_{i=1}^3\sum_{j=1}^3 \Im\bigl(U_{i,j}\bigr)$
% should be zero. The problem is that, if we consider as an example
% \[
%  U=\frac{1}{\sqrt{2}}\,\begin{pmatrix}
%     1&\imath&0\\
%     \imath&1&0\\
%     0&0&\sqrt{2}
%    \end{pmatrix}\;,
% \]
% then
% \[
%  \sum_{i=1}^3\sum_{j=1}^3 \Im\bigl(U_{i,j}\bigr)=
%  \frac{1}{\sqrt{2}} \sum_{i=1}^3\sum_{j=1}^3 \begin{pmatrix}
%     0&1&0\\
%     1&0&0\\
%     0&0&0
%    \end{pmatrix}_{i,j}=\sqrt{2}\neq0.
% \]








\end{document}
