\documentclass[twoside,a4paper]{article}

\usepackage[utf8]{inputenc}
\usepackage[american,english]{babel}
\usepackage{tikz-uml}
\usepackage{listings}
\lstset{language=C++,tabsize=2}
\lstset{numbers=left, stepnumber=1, numbersep=5pt}
\usepackage{units}
\usepackage{rotating}
\usepackage{todonotes}

\title{\Optimal Architecture Overview}
\author{Matthias Bach}

\begin{document}

\newcommand{\Optimal}{OpTiMaL~}
\newcommand{\OpenCL}[0]{OpenCL\texttrademark~}
\newcommand{\class}[1]{\emph{#1}}
\newcommand{\package}[1]{\emph{#1}}

\maketitle

\tableofcontents

\section{Introduction}

This document describes the next version of the \Optimal application architecture.
While it imposes a major refactoring it is an evolution from the original architecture, focusing on cleaner separation of concepts, while retaining parts of the old architecture, like the \OpenCL modules.
Those however now only exist to generate the \OpenCL code used by the other modules.
Especially they are no longer responsible for managing the buffers used by the application and they no longer contain algorithms that operate on the data.

\section{Overview}

The code is now split into several packages.
Each package represents a distinct level ob abstraction. Before going into any details look at the most high level grouping of the code.

The package \package{Physics} contains everything required to describe the program logic.
It is itself contains the sub packages \package{Lattices} and \package{Algorithms}.
The first contains classes representing lattices of different type, e.g. a complete field of spinors, the latter contains algorithms that work on these objects, e.g. an Inverter.

The package \package{Hardware} contains all code that is responsible for matching the logic of the application onto the actual hardware.
Of this only the \class{System} class should be directly used by the application.
Anything else is to actually implemented the types from the \package{Physics} package on the given system.
Type safe \class{Buffer} classes are given that can be used by the classes in \package{Lattices} to store their data on the given hardware in proper formats.
\todo[inline]{Buffer might actually be a template of which specializations exist for each type...}
Note here, that the mapping does not have to be one-to-one, but a lattice class might store its data in multiple buffers, e.g. on multiple devices.

The package \package{Meta} contains things that neither fit well into the \package{Physics} not into the \package{Hardware} package.
For example the parsing and representation of input parameters is located here.
Members of this package are expected to be used by the top level application as well as the \package{Physics} and the \package{Hardware} package.

\begin{sidewaysfigure}
\begin{tikzpicture}
 
\begin{umlpackage}[]{Physics} 

\begin{umlpackage}[y=10]{Lattices} 

\umlemptyclass[x=0,y=2]{Gaugefield}

\umlemptyclass[x=0,y=0]{Gaugemomenta}

\umlemptyclass[x=3,y=0]{Spinorfield}

\umlemptyclass[x=3,y=2]{Spinorfield\_eo}

\end{umlpackage}

\begin{umlpackage}[y=0]{Algorithms} 

\umlemptyclass[x=0,y=2]{Heatbath}

\umlemptyclass[x=0,y=0]{Hmc}

\umlemptyclass[x=3,y=0]{Inverter}

\end{umlpackage}

\umlemptyclass[y=6]{PRNG}

\end{umlpackage}



\begin{umlpackage}[x=9, y=0]{Hardware} 

\begin{umlpackage}[y=7]{Buffers} 

\umlemptyclass[y=6]{Buffer}

\umlemptyclass[y=4]{floatBuffer}

\umlemptyclass[x=3, y=4]{complexBuffer}

\umlemptyclass[y=2]{SU3Buffer}

\umlemptyclass[x=3, y=2]{spinorBuffer}

\umlemptyclass[y=0]{PRNGBuffer}

\end{umlpackage}



\begin{umlpackage}[]{Code} 

\umlemptyclass[y=4]{OpenclModule}

\umlemptyclass[y=4, x=3]{PRNG}

\umlemptyclass[y=2]{Heatbath}

\umlemptyclass[y=2, x=3]{Spinors}

\umlemptyclass[y=0]{Fermions}

\umlemptyclass[y=0, x=3]{Hmc}


\end{umlpackage}

\end{umlpackage}

\begin{umlpackage}[x=18, y=9]{Hardware} 

\umlemptyclass[y=2]{System}

\umlemptyclass[y=0]{Device}

\end{umlpackage}

\begin{umlpackage}[x=18, y=0]{Meta}

\umlemptyclass{Inputparameters}

\end{umlpackage}

\end{tikzpicture}

\caption{High level class diagram of the \Optimal architecture. Be aware of the fact that the \package{Hardware} package has been split for layout reasons.}
\label{cdOverview}

\end{sidewaysfigure}

\todo{The OpenclModules might be required to be refactored to match used datatypes}

All of this objects are to be seen as resource-handlers. Therefore none of them should be copyable.

How all those objects interact shall be explained using the example of the Inverter.
An application will first create an \class{Inputparameters} object. It will then instantiate \class{System}, which will require the \class{Input parameters} object for its initialization.
Afterwards the application can create \class{Gaugefield}s and \class{Spinorfield}s.
These will again require the \class{System} instance for initialization. To perform the inversion the application will now create an \class{Inverter} object.
Given a set of \class{Gaugefield}s and \class{Spinorfields}s this will perform the inversion.
Note that the inversion is actually a solve. \todo{Might be better to have CG and BiCG classes.}

The previous paragraph described the object interaction as it is observer from the top level application.
The implementations of these objects add further interactions.
Creation of objects from the \package{Lattices} package will cause the newly created objects to create \class{Buffer} instances on the devices described by the \class{Device} objects of the given \class{System} instance.
In addition, creation and invocation of the \class{Inverter} object may cause the creation of additional objects from either the \package{Lattices} or \package{Buffer} package.
Note however, that the latter should ideally be wrapped by some class living in the \package{Physics} package.
Further, all objects in the \package{Physics} package will create objects from the \package{Code} package to perform their on device operations.\todo{It might make sense to somehow bundle this code generation to not instantiate kernels to often. Maybe specify required code objects on system creation.}

\listoftodos

\end{document}