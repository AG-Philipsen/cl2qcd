% Copyright (c) 2013,2018 Alessandro Sciarra
%
% This file is part of CL2QCD.
%
% CL2QCD is free software: you can redistribute it and/or modify
% it under the terms of the GNU General Public License as published by
% the Free Software Foundation, either version 3 of the License, or
% (at your option) any later version.
%
% CL2QCD is distributed in the hope that it will be useful,
% but WITHOUT ANY WARRANTY; without even the implied warranty of
% MERCHANTABILITY or FITNESS FOR A PARTICULAR PURPOSE. See the
% GNU General Public License for more details.
%
% You should have received a copy of the GNU General Public License
% along with CL2QCD. If not, see <http://www.gnu.org/licenses/>.

\documentclass[a4paper,10pt]{article}
\usepackage[utf8]{inputenc}
\usepackage[english]{babel}
\usepackage[dvipsnames,usenames]{xcolor}
\usepackage{geometry}
\usepackage{indentfirst}
\usepackage{amsmath,amssymb,bbm,mathabx}
\usepackage{slashed}
\usepackage{microtype}
\usepackage{braket}
\usepackage{simplewick}
\usepackage{graphicx}
\usepackage{cleveref}
\usepackage{cancel}
\usepackage{nccmath} %For fleqn environment

%geometry
\geometry{textheight=23cm,textwidth=16cm,hmarginratio=1:1,vmarginratio=1:1}
%cleveref
\crefname{equation}{Eq.}{Eqs.}

%Commands
\newcommand{\Eq}[1]{Eq.~\eqref{eq:#1}}
\newcommand{\Eqs}[1]{Eqs.~\eqref{eq:#1}}

%opening
\title{\vspace*{-2cm}\huge{\textbf{\texttt{ae} object from su(3) matrix}}\vspace{-0.2cm}}
\author{\texttt{of} \emph{Alessandro Sciarra}}
\date{}

\begin{document}

\maketitle

It is well known that whatever elements $\Omega\in SU(3)$ can be written as
\begin{equation}\label{eq:1}
 \Omega=\exp\Biggl(\imath\sum_{k=1}^8 a^k\, T_k\Biggr)\;,
\end{equation}
where $a^k$ are the 8 \emph{real} numbers needed to parameterize $\Omega$.
The generators $T_k$ are chosen as traceless, complex, and hermitian $3\times3$
matrices obeying the normalization condition
\begin{equation}\label{eq:2}
 Tr[T_j\,T_k]=\frac{1}{2}\:\delta_{jk}\;.
\end{equation}
The standard representation of the generators for $SU(3)$ is given by
\begin{equation}\label{eq:3}
 T_k=\frac{\lambda_k}{2}\;,
\end{equation}
where $\lambda_k$ are the so called \emph{Gell-Mann matrices}:
\begin{alignat*}{4}
 \lambda_1 &= \left(
	      \begin{array}{ccc}
	      0 & 1 & 0 \\
	      1 & 0 & 0 \\
	      0 & 0 & 0
	      \end{array}
	      \right) \qquad&
 \lambda_2 &= \left(
	      \begin{array}{ccc}
	       0 & -i & 0 \\
	       i & 0 & 0 \\
	       0 & 0 & 0
	       \end{array}
	       \right) \qquad&
 \lambda_3 &= \left(
	      \begin{array}{ccc}
	      1 & 0 & 0 \\
	      0 & -1 & 0 \\
	      0 & 0 & 0
	      \end{array}
	      \right) \\[1ex]
 \lambda_4 &= \left(
	      \begin{array}{ccc}
	      0 & 0 & 1 \\
	      0 & 0 & 0 \\
	      1 & 0 & 0
	      \end{array}
	      \right) \qquad&
 \lambda_5 &= \left(
	      \begin{array}{ccc}
	      0 & 0 & -i \\
	      0 & 0 & 0 \\
	      i & 0 & 0
	      \end{array}
	      \right) \qquad&
 \lambda_6 &= \left(
	      \begin{array}{ccc}
	      0 & 0 & 0 \\
	      0 & 0 & 1 \\
	      0 & 1 & 0
	      \end{array}
	      \right) \\[1ex]
 \lambda_7 &= \left(
	      \begin{array}{ccc}
	      0 & 0 & 0 \\
	      0 & 0 & -i \\
	      0 & i & 0
	      \end{array}
	      \right) \qquad&
 \lambda_8 &= \frac{1}{\sqrt{3}}\left(
	      \begin{array}{ccc}
	      1 & 0 & 0 \\
	      0 & 1 & 0 \\
	      0 & 0 & -2
	      \end{array}
	      \right)\;.
\end{alignat*}

The problem we want to address now is the following: given a generic $su(3)$ matrix
how can we reconstruct the 8 real coefficients $a^k$ of \Eq{1}? It is basically
an algebra calculation. Let us observe that a generic $su(3)$ matrix $M$ has the form
\begin{equation}\label{eq:4}
 M=\left(
  \begin{array}{ccc}
   m_1 & m_2+\imath\: m_3 & m_4+\imath\: m_5 \\
   m_2-\imath\: m_3 & m_6 & m_7+\imath\: m_8 \\
   m_4-\imath\: m_5 & m_7-\imath\: m_8 & -m_1-m_6
  \end{array}
 \right)\;,
\end{equation}
since it has to be hermitian and traceless ($m_i\in\mathbb{R}$). Moreover, it is straightforward
to check that
\begin{equation}\label{eq:5}
 \sum_{k=1}^8 a^k\, T_k = \frac{1}{2}
    \left(
      \begin{array}{ccc}
        a_3+\frac{a_8}{\sqrt{3}} & a_1-i a_2 & a_4-i a_5 \\
	a_1+i a_2 & \frac{a_8}{\sqrt{3}}-a_3 & a_6-i a_7 \\
	a_4+i a_5 & a_6+i a_7 & -\frac{2 a_8}{\sqrt{3}}
      \end{array}
    \right)\;;
\end{equation}
hence, to answer our question, we have only to set the r.h.s. of \Eq{4} equal to the r.h.s. of \Eq{5}.
Taking into account that  we are dealing with hermitian, traceless matrices, we have then only 5 equations,
whose some are complex:
\begin{equation}
 \left\{
  \begin{aligned}
    a_1-i\:a_2 &= 2\:(m_2+\imath\: m_3) \\
    a_4-i\:a_5 &= 2\:(m_4+\imath\: m_5) \\
    a_8-\sqrt{3}\:a_3 &= 2\sqrt{3}\: m_6 \\
    a_6-i\:a_7 &= 2\:(m_7+\imath\: m_8) \\
    -2\:a_8 &= - 2\sqrt{3}\:(m_1+m_6)
  \end{aligned}
 \right.\;.
\end{equation}
Solving the system, above we obtain:
\begin{equation}
 \left\{
  \begin{aligned}
    a_1 &= 2\:m_2 \\
    a_2 &= -2\:m_3 \\
    a_3 &= m_1-m_6 \\
    a_4 &= 2\:m_4 \\
    a_5 &= -2\:m_5 \\
    a_6 &= 2\:m_7 \\
    a_7 &= -2\:m_8 \\
    a_8 &= \sqrt{3}\:(m_1+m_6)
  \end{aligned}
 \right.
\end{equation}
that conludes our calculation.


\end{document}
